\documentclass[preview]{standalone}
\usepackage[english]{babel}
\usepackage{amsmath}
\usepackage{amssymb}
\begin{document}
\begin{center}
\textbf{Complexidade Total:} $O((V + E) \cdot \log V)$
            \begin{itemize}
                \item A complexidade total é a soma das complexidades das operações:
                    \begin{itemize}
                        \item Inicialização: $O(V)$
                        \item Extrair Mínimo: $O(V \cdot \log V)$
                        \item Relaxamento de Arestas: $O(E \cdot \log V)$
                    \end{itemize}
                \item O termo dominante depende da estrutura do grafo:
                    \begin{itemize}
                        \item Para grafos esparsos ($E \approx V$), a complexidade é $O(V \cdot \log V)$.
                        \item Para grafos densos ($E \approx V^2$), a complexidade é $O(E \cdot \log V)$.
                    \end{itemize}
            \end{itemize}
            
            \textbf{Conclusão:} A complexidade total é $O((V + E) \cdot \log V)$, combinando o custo da inicialização, extração mínima e relaxamento de arestas.
\end{center}
\end{document}